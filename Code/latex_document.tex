\documentclass[12pt,a4]{article}
\usepackage{graphicx} % Required for inserting images
\usepackage{hyperref} % to insert hyperlinks of references
\usepackage{lineno}
\usepackage{natbib}
\linenumbers

\title{spatial ecology in R test}
\author{jonahme }
\date{January 2026} % to exclude date: delete argument, not function, or remove to get date from computer

\begin{document}

\maketitle

\tableofcontents

\section{Introduction} \label{sec:intro}
Lorem ipsum
 
dolor sit amet

Über die Herkunft des Textes lässt sich keine Klarheit mehr gewinnen. Auch finden sich viele leicht voneinander abweichende Varianten dieses Textes. In der 1914 herausgegebenen lateinisch-englischen Cicero-Werkausgabe findet sich der Text auf Seite 36 oben beginnend mit lorem ipsum…, da das Wort dolorem getrennt umbrochen wurde

\smallskip 
\noindent Über die Herkunft des Textes lässt sich keine Klarheit mehr gewinnen. Auch finden sich viele leicht voneinander abweichende Varianten dieses Textes. In der 1914 herausgegebenen lateinisch-englischen Cicero-Werkausgabe findet sich der Text auf Seite 36 oben beginnend mit lorem ipsum…, da das Wort dolorem getrennt umbrochen wurde

\bigskip
Über die Herkunft des Textes lässt sich keine Klarheit mehr gewinnen. Auch finden sich viele leicht voneinander abweichende Varianten dieses Textes. In der 1914 herausgegebenen lateinisch-englischen Cicero-Werkausgabe findet sich der Text auf Seite 36 oben beginnend mit lorem ipsum…, da das Wort dolorem getrennt umbrochen wurde

\subsection{aim}
blabla

\section{Methods} \label{sec:methods}
In this thesis I made use of the function related to gravity as in equation \ref{eq:newton}

\begin{equation}
    F= G \times \frac{m_1 \times m_2}{r^2}
    \label{eq:newton}
\end{equation}

The above equation was blabla equation \ref{eq:complex}

\begin{equation}
    F= \sqrt[3]{G \times \frac{m_1 \times m_2}{r^2}}
    \label{eq:complex}
\end{equation}

As in section \ref{sec:intro}

\section{Results}
\section{Discussion}

Uaing the formulas presented in section \ref{sec:methods} I obtained the results in figure \ref{fig:result}.

\begin{figure}
    \centering
    \includegraphics[width=0.8\linewidth]{Bildschirmfoto 2024-07-31 um 15.46.29.png}
    \caption{blabla}
    \label{fig:result}
\end{figure}

My results were attained by \citep{brodrick,lek}.

\begin{thebibliography}{999}
    \bibitem[Brodrick et al.(2019)]{brodrick}
    Brodrick, P. G., Davies, A. B., \& Asner, G. P. (2019). Uncovering ecological patterns with convolutional neural networks. Trends in ecology \& evolution, 34(8), 734-745.

    \bibitem[Lek et al.(2012)]{lek}
    Lek, S., \& Guégan, J. F. (Eds.). (2012). Artificial neuronal networks: application to ecology and evolution. Springer Science \& Business Media.
\end{thebibliography}

\section*{Acknowledgement}

\end{document}
